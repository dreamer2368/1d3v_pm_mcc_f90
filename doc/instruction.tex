\documentclass[11pt]{article}

\usepackage[dvips]{graphicx}
\usepackage{multicol}
\usepackage{float}
\usepackage{psfrag}
\usepackage{amsmath,amssymb,rotating,dcolumn,texdraw,tabularx,colordvi}
\usepackage[usenames]{color}
\usepackage[nooneline,tight,raggedright]{subfigure}
\usepackage{amsthm}

\usepackage{wrapfig}
\usepackage{pstricks,enumerate}

\usepackage[font=footnotesize,format=plain,labelfont=bf]{caption}

\usepackage{hyperref}

%\captionsetup{labelfont={color=Brown,bf},textfont={color=BurntOrange}}

\definecolor{myTan}{rgb}{.7,0.4,.15}
\captionsetup{labelfont={color=brown,bf},textfont={color=myTan}}

\newcommand{\entry}[1]{\mbox{\sffamily\bfseries{#1:}}\hfil}%

\setlength{\marginparwidth}{.65in}
\def\margcomment#1{\Red{$\bullet$}\marginpar{\raggedright \Red{\tiny #1}}}

\makeatletter
\renewcommand{\section}{\@startsection
{section}%
{0}%
{0mm}%
{-0.35\baselineskip}%
{0.01\baselineskip}%
{\normalfont\Large\bfseries\color{brown}}}%
\makeatother

\makeatletter
\renewcommand{\subsection}{\@startsection
{subsection}%
{1}%
{0mm}%
{-0.35\baselineskip}%
{0.1\baselineskip}%
{\normalfont\large\bfseries\color{brown}}}%
\makeatother


\makeatletter
\renewcommand{\subsubsection}{\@startsection
{subsubsection}%
{1}%
{0mm}%
{-0.5\baselineskip}%
{0.3\baselineskip}%
{\normalfont\normalsize\itshape\centering\color{brown}}}%
\makeatother

%\renewcommand{\topfraction}{0.0}
\renewcommand{\textfraction}{0.0}
\renewcommand{\floatpagefraction}{0.7}


\setlength{\oddsidemargin}{0.0in}
\setlength{\textwidth}{6.5in}
\setlength{\topmargin}{-0.5in}
\setlength{\footskip}{0.30in}
\setlength{\textheight}{9.0in}
\setlength{\headheight}{0.2in}
\setlength{\headsep}{0.3in}

\def\Dpartial#1#2{ \frac{\partial #1}{\partial #2} }
\def\Dparttwo#1#2{ \frac{\partial^2 #1 }{ \partial #2^2} }
\def\Dpartpart#1#2#3{ \frac{\partial^2 #1}{ \partial #2 \partial #3} }
\def\Dnorm#1#2{ \frac{d #1 }{ d #2} }
\def\Dnormtwo#1#2{ \frac{d^2 #1}{  d #2 ^2} }
\def\Dtotal#1#2{ \frac{D #1 }{ D #2} }
\def\Del#1#2{ \frac{\delta #1}{\delta #2} }
\def\Var#1{\Dnorm{}{\epsilon} #1 \bigg|_{\epsilon=0}}
\def\inprod#1#2{ \left\langle #1 , #2 \right\rangle}

\def\eps{\varepsilon}

\newcommand{\vp}{v_p}
\newcommand{\xp}{x_p}
\newcommand{\vps}{v_p^*}
\newcommand{\xps}{x_p^*}
\newcommand{\Es}{E^*}
\newcommand{\phis}{\phi^*}
\newcommand{\Dx}{\Delta x}
\newcommand{\Dt}{\Delta t}

\newcommand{\vph}{\hat{v}_p}
\newcommand{\xph}{\hat{x}_p}
\newcommand{\Eh}{\hat{E}}
\newcommand{\phih}{\hat{\phi}}
\newcommand{\rhoh}{\hat{\rho}}

\newcommand{\cH}{\mathcal{H}}
\newcommand{\cJ}{\mathcal{J}}
\newcommand{\cL}{\mathcal{L}}
\newcommand{\cD}{\mathcal{D}}

\newcommand{\timesum}{\sum\limits_{n=0}^{N_t-1}}
\newcommand{\particlesum}{\sum\limits_{p=1}^{N}}
\newcommand{\meshsum}{\sum\limits_{i=1}^{N_g}}

\newcommand{\bxi}{\boldsymbol{\xi}}

\newcommand{\bbK}{\hat{\mathbb{K}}}

\newcommand{\myint}{\int_0^{T}\sum\limits_{p=1}^N}
\newcommand{\mysum}{\sum\limits_{p=1}^N\int_0^{T}}
\newcommand{\myiint}{\int_0^{T}\int_0^L}
\newcommand{\dt}{\; dt}

\def\bdash{\hbox{\drawline{4}{.5}\spacce{2}}}
\def\spacce#1{\hskip #1pt}
\def\drawline#1#2{\raise 2.5pt\vAox{\hrule width #1pt height #2pt}}
\def\dashed{\bdash\bdash\bdash\bdash\nobreak\ }
\def\solid{\drawline{24}{.5}\nobreak\ }
\def\square{${\vcenter{\hrule height .4pt 
              \hbox{\vrule width .4pt height 3pt \kern 3pt \vrule width .4pt}
          \hrule height .4pt}}$\nobreak\ }
\def\solidsquare{${\vcenter{\hrule height 3pt width 3pt}}$\nobreak\ }


\renewcommand{\thefootnote}{\fnsymbol{footnote}}

 \renewcommand{\topfraction}{0.9}
    \renewcommand{\bottomfraction}{0.8}	
    \setcounter{topnumber}{2}
    \setcounter{bottomnumber}{2}
    \setcounter{totalnumber}{4}     % 2 may work better
    \setcounter{dbltopnumber}{2}    % for 2-column pages
    \renewcommand{\dbltopfraction}{0.9}	% fit big float above 2-col. text
    \renewcommand{\textfraction}{0.07}	% allow minimal text w. figs
    \renewcommand{\floatpagefraction}{0.7}	% require fuller float pages
    \renewcommand{\dblfloatpagefraction}{0.7}	% require fuller float pages

\setlength{\parindent}{0.25in}
\setlength{\parskip}{2.0ex}

\newtheorem*{remark}{Remark}

\title{Instructions on 1d3v\_pm\_mcc\_f90 code}
\author{Seung Whan Chung}

%%%%%%%%%%%%%%%%%%%%%%%%%%%%%%%%%%%%%%%%%%%%%
\begin{document}
\maketitle
\tableofcontents
\clearpage

\subsection*{Continuous governing equation}
Following equation is the governing equation for particle-in-cell methods,
continuous in both time and space.
\begin{equation*}
\begin{split}
\Dnorm{\xp}{t} - \vp &= 0\\
\rho - \sum_p f(x,\xp) &= 0\\
\nabla^2\phi + \frac{\rho}{\varepsilon_0} &= 0\\
E + \nabla\phi &= 0\\
E_p - \int E(x')g(x'-\xp)\;dx' &= 0\\
\Dnorm{\vp}{t} - \frac{q}{m}E_p &= 0
\end{split}
\end{equation*}
$f(\cdot,\cdot)$ and $g(\cdot,\cdot)$ correspond to the interpolation from particle to grid and vice versa, respectively.

\subsection*{Continuous constraint functional}
For continuous governing equation, we have following constraint functional $\cH$ (or Lagrangian)
\begin{equation*}
\begin{split}
\cH :=& \langle \hat{Q}, \mathcal{M}[Q] \rangle\\
=& \int_0^T\sum_{p=1}^{N} \xph\left[ \Dnorm{\xp}{t} - \vp \right]dt\\
&+ \int_0^T\int_X \rhoh\left[ \rho - \sum_p f(x,\xp) \right]dxdt\\
&+ \int_0^T\int_X \phih\left[ \nabla^2\phi + \frac{\rho}{\varepsilon_0} \right]dxdt\\
&+ \int_0^T\int_X \Eh\cdot\left[ E + \nabla\phi \right]dxdt\\
&+ \int_0^T\sum_{p=1}^{N} \Eh_p\left[ E_p - \int E(x')g(x'-\xp)\;dx' \right]dt\\
&+ \int_0^T\sum_{p=1}^{N} \vph\left[ \Dnorm{\vp}{t} - \frac{q}{m}E_p \right]dt,
\end{split}
\end{equation*}
where 6 integral terms correspond to sub-inner-products with respect to which we are supposed to find sub-adjoint operators.
Adjoint formulation for other operators can be trivial, here we formulate the sub-adjoint operator for time-derivative,
for the purpose of comparison with discrete case.
\begin{equation*}
\int_0^T \xph\Dnorm{\delta\xp}{t}\;dt = - \int_0^T \Dnorm{\xph}{t}\delta\xp\;dt + \xph\delta\xp\bigg|_0^T,
\end{equation*}
where $\xph\delta\xp\bigg|_T$ is used to set final condition for adjoint variable,
and $\xph\delta\xp\bigg|_0$ is used to compute sensitivity to initial condition.

\subsection*{Discrete governing equation}
$n=1,\cdots,N_t$
\begin{equation*}
\begin{split}
\frac{\xp^{n} - \xp^{n-1}}{\Dt} - \vp^{n-1/2} &= 0\\
\rho_{i}^n - \sum\limits_p f\left( x_{i}, x_p^n \right) &= 0\\
\phi_i^n + K^{-1}\frac{\rho_i^n}{\varepsilon_0} &= 0\\
E_i^n + D\phi_i^n &= 0\\
E_{p}^{n} - \sum\limits_{i}E_{i}^{n}\cdot g\left( x_{i}, x_p^{n} \right) &= 0\\
\frac{\vp^{n+1/2} - \vp^{n-1/2}}{\Dt} - \frac{q_p}{m_p}E_p^{n} &= 0
\end{split}
\end{equation*}
where $K^{-1}$, $D$ are spatial-discretized operators of $(\nabla^2)^{-1}$ and $\nabla\cdot$, respectively.

\subsection*{Discrete constraint functional}
We have constraint functional analogous to continuous case.
\begin{equation*}
\begin{split}
\cH :=& \langle \hat{Q}, \mathcal{M}[Q] \rangle\\
&+ \sum_{n=1}^{N_T}\sum_{p=1}^{N}\xph^{n}\left[ \frac{\xp^{n} - \xp^{n-1}}{\Dt} - \vp^{n-1/2} \right]\\
&+ \sum_{n=1}^{N_T}\sum_{i=1}^{N_g}\rhoh_{i}^n\left[ \rho_{i}^n - \sum\limits_p f\left( x_{i}, x_p^n \right) \right]\\
&+ \sum_{n=1}^{N_T}\sum_{i=1}^{N_g}\left( \phih_i^{n} \right)^T\left[ \phi_i^n + K^{-1}\frac{\rho_i^n}{\varepsilon_0} \right]\\
&+ \sum_{n=1}^{N_T}\sum_{i=1}^{N_g}\left( \Eh_i^n \right)^T\left[ E_i^n + D\phi_i^n \right]\\
&+ \sum_{n=1}^{N_T}\sum_{p=1}^{N}\Eh_{p}^{n}\left[ E_{p}^{n} - \sum\limits_{i}E_{i}^{n}\cdot g\left( x_{i}, x_p^{n} \right) \right]\\
&+ \sum_{n=1}^{N_T}\sum_{p=1}^{N}\vph^{n+1/2}\left[ \frac{\vp^{n+1/2} - \vp^{n-1/2}}{\Dt} - \frac{q_p}{m_p}E_p^{n} \right]
\end{split}
\end{equation*}
Technically, we are supposed to add quadrature weights corresponding to $\Delta x$ and $\Delta t$.
But we didn't, and as a result we have $1/\Delta t$ factors in sensitivity term to initial condition.
Furthermore, at this point the inner product and the time derivative are not defined such that
the time derivative operator would satisfy SBP property.
In fact, while $\xph$ have the size $N_T$ in timestep $n=1,\cdots,N_T$,
$\xp$ have the size $N_T+1$ with timestep $n=0,\cdots,N_T$.
This makes the time derivative operator not even a square matrix.
This requires us to carefully transpose this matrix and handle final and initial conditions.
\clearpage

\subsection*{Adjoints blocks}
\addcontentsline{toc}{subsection}{Adjoint blocks}

\begin{multicols}{2}
$\cH$ blocks : $n=1,\cdots,N_t$
\begin{equation*}
\begin{split}
\xph^{n}\left[ \frac{\xp^{n} - \xp^{n-1}}{\Dt} - \vp^{n-1/2} \right] &= 0\\
\rhoh_{i}^n\left[ \rho_{i}^n - \sum\limits_p f\left( x_{i}, x_p^n \right) \right] &= 0\\
\left( \phih_i^{n} \right)^T\left[ \phi_i^n + K^{-1}\frac{\rho_i^n}{\varepsilon_0} \right] &= 0\\
\left( \Eh_i^n \right)^T\left[ E_i^n + D\phi_i^n \right] &= 0\\
\Eh_{p}^{n}\left[ E_{p}^{n} - \sum\limits_{i}E_{i}^{n}\cdot g\left( x_{i}, x_p^{n} \right) \right] &= 0\\
\vph^{n+1/2}\left[ \frac{\vp^{n+1/2} - \vp^{n-1/2}}{\Dt} - \frac{q_p}{m_p}E_p^{n} \right] &= 0
\end{split}
\end{equation*}
Adjoint equation : $n=1,\cdots,N_t$
\begin{equation*}
\begin{split}
\frac{\vph^{n+3/2} - \vph^{n+1/2}}{-\Dt} - \xph^{n+1} &= 0\qquad:\delta \vp^{n+1/2}\\
\Eh_{p}^{n} - \frac{q_p}{m_p}\vph^{n+1/2} &= 0\qquad:\delta E_p^{n}\\
\Eh_{i}^{n} - \sum\limits_p \Eh_{p}^{n}\cdot g\left( x_{i}, x_p^{n} \right) &= 0\qquad:\delta E_i^{n}\\
\phih_i^{n} - D\Eh_i^{n} &= 0\qquad:\delta \phi_i^{n}\\
\rhoh_i^{n} + \frac{1}{\varepsilon_0}K^{-1}\phih_i^{n} &= 0\qquad:\delta \rho_i^{n}\\
\frac{\xph^{n+1} - \xph^{n}}{-\Dt} + \sum\limits_{i}\rhoh_{i}^{n}\cdot\left( -\Dpartial{f}{x_p^{n}} \right) + &\\
\qquad\sum\limits_{i} \Eh_{p}^{n}E_{i}^{n}\cdot\left( -\Dpartial{g}{x_p^{n}} \right) &= 0\qquad:\delta \xp^{n}
\end{split}
\end{equation*}
\end{multicols}
$n=0,\cdots,N_t-1$ timesteps are saved.
\begin{itemize}
\item Adjoint in initial timestep ($n=0$)
\begin{equation}
\frac{\xph^1-\xph^0}{-\Dt}\delta \xp^0 = 0\qquad\qquad \left[ \frac{\vph^{3/2} - \vph^{1/2}}{-\Dt} - \xph^1 \right]\delta \vp^{1/2}=0
\end{equation}
You can just use $\xph^1$, $\vph^{3/2}$ for sensitivity,
but for the case we have initial conditions in $\cJ$, this computation for initial step is included in this code.
\item Adjoint final condition
\begin{equation}
\frac{\vph^{N_t+3/2}}{\Dt}\delta v_p^{N_t+1/2} = 0\qquad\qquad\frac{\xph^{N_t+1}}{\Dt}\delta x_p^{N_t} = 0
\end{equation}
\item Sensitivity to initial condition (add to $\delta \cJ$)
\begin{equation}
-\frac{\xph^0}{\Dt}\delta \xp^0 - \frac{\vph^{1/2}}{\Dt}\delta \vp^{1/2}
\end{equation}
\end{itemize}

\end{document}